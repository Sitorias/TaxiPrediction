%%%%%%%%%%%%%%%%%%%%%%%%%%%%%%%%%%%%%%%%%%%%%%%%%%%%%%%%%%%%%%%%%%%%%%
% LaTeX Template: Beamer arrows
%
% Source: http://www.texample.net/
% Feel free to distribute this template, but please keep the
% referal to TeXample.net.
% Date: Nov 2006
% 
%%%%%%%%%%%%%%%%%%%%%%%%%%%%%%%%%%%%%%%%%%%%%%%%%%%%%%%%%%%%%%%%%%%%%%
% How to use writeLaTeX: 
%
% You edit the source code here on the left, and the preview on the
% right shows you the result within a few seconds.
%
% Bookmark this page and share the URL with your co-authors. They can
% edit at the same time!
%
% You can upload figures, bibliographies, custom classes and
% styles using the files menu.
%
% If you're new to LaTeX, the wikibook is a great place to start:
% http://en.wikibooks.org/wiki/LaTeX
%
%%%%%%%%%%%%%%%%%%%%%%%%%%%%%%%%%%%%%%%%%%%%%%%%%%%%%%%%%%%%%%%%%%%%%%

\documentclass{beamer} %
\usetheme{CambridgeUS}
\usepackage[latin1]{inputenc}
\usefonttheme{professionalfonts}
\usepackage{times}
\usepackage{tikz}
\usepackage{amsmath}
\usepackage{verbatim}
\usetikzlibrary{arrows,shapes}

\author{Leonoard Schmischke, Sven Durchholz}
\title{Taxi Prediction}

\begin{document}

\frame{\titlepage}


\begin{comment}
:Title: Beamer arrows
:Tags: Remember picture, Beamer, Physics & chemistry, Overlays
:Use page: 3

With PGF/TikZ version 1.09 and later, it is possible to draw paths between nodes across
different pictures. This is a useful feature for presentations with the
Beamer package. In this example I've combined the new PGF/TikZ's overlay feature
with Beamer overlays. Download the PDF version to see the result.

**Note.** This only works with PDFTeX, and you have to run PDFTeX twice.

| Author: Kjell Magne Fauske

\end{comment}


% For every picture that defines or uses external nodes, you'll have to
% apply the 'remember picture' style. To avoid some typing, we'll apply
% the style to all pictures.
\tikzstyle{every picture}+=[remember picture]

% By default all math in TikZ nodes are set in inline mode. Change this to
% displaystyle so that we don't get small fractions.
\everymath{\displaystyle}

\begin{frame}
\frametitle{Data Set}
\begin{itemize}
	\item 5 Ratecodes
	\begin{enumerate}
		\item Metered
		\item JFK
		\item Newark
		\item Out of town
		\item Negoiated
		\item Group
	\end{enumerate}
\end{itemize}
\end{frame}

\begin{frame}
\frametitle{Data Cleaning}
\end{frame}

\begin{frame}
\frametitle{Exploration}
\end{frame}

\begin{frame}
\frametitle{Model concept}
\end{frame}

\begin{frame}
\frametitle{Extras}
\end{frame}

\begin{frame}
\frametitle{MTA}
\end{frame}

\begin{frame}
\frametitle{Fare amount}
\end{frame}

\begin{frame}
\frametitle{Tips}
\end{frame}

\begin{frame}
\frametitle{Putting togehter}
\end{frame}

\begin{frame}
\frametitle{Conclusion}
\end{frame}
\end{document}